\documentclass[]{article}
\usepackage{lmodern}
\usepackage{amssymb,amsmath}
\usepackage{ifxetex,ifluatex}
\usepackage{fixltx2e} % provides \textsubscript
\ifnum 0\ifxetex 1\fi\ifluatex 1\fi=0 % if pdftex
  \usepackage[T1]{fontenc}
  \usepackage[utf8]{inputenc}
\else % if luatex or xelatex
  \ifxetex
    \usepackage{mathspec}
    \usepackage{xltxtra,xunicode}
  \else
    \usepackage{fontspec}
  \fi
  \defaultfontfeatures{Mapping=tex-text,Scale=MatchLowercase}
  \newcommand{\euro}{€}
\fi
% use upquote if available, for straight quotes in verbatim environments
\IfFileExists{upquote.sty}{\usepackage{upquote}}{}
% use microtype if available
\IfFileExists{microtype.sty}{%
\usepackage{microtype}
\UseMicrotypeSet[protrusion]{basicmath} % disable protrusion for tt fonts
}{}
\ifxetex
  \usepackage[setpagesize=false, % page size defined by xetex
              unicode=false, % unicode breaks when used with xetex
              xetex]{hyperref}
\else
  \usepackage[unicode=true]{hyperref}
\fi
\usepackage[usenames,dvipsnames]{color}
\hypersetup{breaklinks=true,
            bookmarks=true,
            pdfauthor={},
            pdftitle={},
            colorlinks=true,
            citecolor=blue,
            urlcolor=blue,
            linkcolor=magenta,
            pdfborder={0 0 0}}
\urlstyle{same}  % don't use monospace font for urls
\setlength{\parindent}{0pt}
\setlength{\parskip}{6pt plus 2pt minus 1pt}
\setlength{\emergencystretch}{3em}  % prevent overfull lines
\providecommand{\tightlist}{%
  \setlength{\itemsep}{0pt}\setlength{\parskip}{0pt}}
\setcounter{secnumdepth}{0}

\date{}

% Redefines (sub)paragraphs to behave more like sections
\ifx\paragraph\undefined\else
\let\oldparagraph\paragraph
\renewcommand{\paragraph}[1]{\oldparagraph{#1}\mbox{}}
\fi
\ifx\subparagraph\undefined\else
\let\oldsubparagraph\subparagraph
\renewcommand{\subparagraph}[1]{\oldsubparagraph{#1}\mbox{}}
\fi

\begin{document}

Top of Form

Scientific Paper Step 4: Nearly-submittable draft

Congratulations on finishing step 3! You are now into the major work of
the paper, which is converting the outline to a nearly submittable
draft.

\begin{itemize}
\item
  A "nearly submittable draft" is a full paper draft that may not yet
  meet your internal quality standards, but it is something that could
  in theory be submitted for publication without much additional work.
\item
  To be clear - you may want to polish your text more, round out your
  literature review, or make more visually attractive figures. But at
  this stage, you are aiming for "good enough" and should mostly stop
  there. There will be time to polish beyond this level later.
\item
  Note that submittable applies to references and citations at this
  stage as well. You will add formal citations in this stage. These
  citations should be complete enough for submission; you can add notes
  regarding other citations you may want to look up later in case new
  information comes in. ~
\end{itemize}

Step 4 usually takes a little bit longer than Step~3, but~can usually be
completed in within a week. If it takes much longer than that, it's a
sign that Step 4 is premature.~

After finishing this step, it is a good idea to collect feedback before
final polishing.

When answering the questions in this form, use plain, conversational
language - short, simple sentences are preferred!

More information: https://github.com/computron/scientific-paper-flow

* Indicates required question

1.

Email *

Tips for Step 4

The nearly submittable draft is somewhat magical because halfway through
this step, any stress you may have had about the paper will start to
disappear because you know that your paper is going to make it. You're
not finished, but you know you *will* finish, and that's a big win for
motivation. So remember that as you go through this step.

The biggest challenge of step 4 is that~it can be time-consuming. Every
bullet point you put in step 3 needs to be turned into a paragraph with
references. For a single bullet, this might anywhere from 10 minutes to
an hour. And since you had 25 - 40 bullets in your outline, this might
end up being a dozen or more hours of work.

If you start to lose momentum in this step, here are a few tips:

\begin{itemize}
\item
  Remember to consider parallelization. If the task is daunting, split
  this step amongst several co-authors.
\item
  Do not worry about your text being poetry. For now, you don't want
  perfect, you want "submittable" (good enough). You'll polish later to
  your quality standards.
\item
  Instead of getting sidetracked (e.g., looking up new extra citations),
  use notes to remind yourself about other ideas and references to
  follow up on. You can color these notes, use comments, etc. Again, you
  will ~go back later.
\item
  Convert bullet points to text in the following order: Methods,
  Results, Intro, Discussion, Conclusion, Abstract. Once a bullet point
  is submittable, stop and move on.
\item
  Remember - as long as you keep going and making progress, you
  \emph{will} finish!
\end{itemize}

2.

\textbf{Please attach your paper which can in principle be
submitted~with only minor changes~to one of your chosen journals (from
Step 1).}

Attach your nearly submittable draft. Please include the Supporting
Information file also if it is present.

\emph{Note that you may still desire to make improvements to the paper,
but this version should essentially be submittable as-is to a journal or
to a preprint server with minor or no further changes.}

Files submitted:

3.

\textbf{What changes (if any) would you want to make to the paper at
this stage?}

Please provide a time estimate for each of these improvements.

\emph{This could be changes you feel would improve the quality, impact,
precision, or academic rigor of results of the paper.}

\emph{{[}max 2000 characters{]}}

4.

\textbf{List ways in which collaborators of the paper may assist with
unfinished items.}

\emph{For example, if your draft has notes on things to further polish,
citations to look up, or additional tests to perform, perhaps a
collaborator can help with this.}

\emph{{[}max: 1000 characters{]}}

*

5.

\textbf{Should this version of the paper now be sent to all
collaborating authors for their review?}

If not, please provide a timeline for when the manuscript will be in
good enough shape to send for internal reviews.

\emph{{[}max 1000 characters{]}}

*

Bottom of Form

\end{document}
