\documentclass[]{article}
\usepackage{lmodern}
\usepackage{amssymb,amsmath}
\usepackage{ifxetex,ifluatex}
\usepackage{fixltx2e} % provides \textsubscript
\ifnum 0\ifxetex 1\fi\ifluatex 1\fi=0 % if pdftex
  \usepackage[T1]{fontenc}
  \usepackage[utf8]{inputenc}
\else % if luatex or xelatex
  \ifxetex
    \usepackage{mathspec}
    \usepackage{xltxtra,xunicode}
  \else
    \usepackage{fontspec}
  \fi
  \defaultfontfeatures{Mapping=tex-text,Scale=MatchLowercase}
  \newcommand{\euro}{€}
\fi
% use upquote if available, for straight quotes in verbatim environments
\IfFileExists{upquote.sty}{\usepackage{upquote}}{}
% use microtype if available
\IfFileExists{microtype.sty}{%
\usepackage{microtype}
\UseMicrotypeSet[protrusion]{basicmath} % disable protrusion for tt fonts
}{}
\ifxetex
  \usepackage[setpagesize=false, % page size defined by xetex
              unicode=false, % unicode breaks when used with xetex
              xetex]{hyperref}
\else
  \usepackage[unicode=true]{hyperref}
\fi
\usepackage[usenames,dvipsnames]{color}
\hypersetup{breaklinks=true,
            bookmarks=true,
            pdfauthor={},
            pdftitle={},
            colorlinks=true,
            citecolor=blue,
            urlcolor=blue,
            linkcolor=magenta,
            pdfborder={0 0 0}}
\urlstyle{same}  % don't use monospace font for urls
\setlength{\parindent}{0pt}
\setlength{\parskip}{6pt plus 2pt minus 1pt}
\setlength{\emergencystretch}{3em}  % prevent overfull lines
\providecommand{\tightlist}{%
  \setlength{\itemsep}{0pt}\setlength{\parskip}{0pt}}
\setcounter{secnumdepth}{0}

\date{}

% Redefines (sub)paragraphs to behave more like sections
\ifx\paragraph\undefined\else
\let\oldparagraph\paragraph
\renewcommand{\paragraph}[1]{\oldparagraph{#1}\mbox{}}
\fi
\ifx\subparagraph\undefined\else
\let\oldsubparagraph\subparagraph
\renewcommand{\subparagraph}[1]{\oldsubparagraph{#1}\mbox{}}
\fi

\begin{document}

Top of Form

Scientific Paper Step 1: Planning

Congratulations on getting started with paper writing!

In step 1, you will think about:

\begin{itemize}
\item
  what is the purpose of the paper, and how might you have even higher
  impact
\item
  who is the audience of the paper
\item
  what might be some potential journals to submit to
\item
  who are your coauthors
\item
  what are some of the key references/citations of your work
\item
  who might be referees of your paper
\item
  what will be the ultimate impacts of you finishing this work?
\end{itemize}

Step 1 is designed to be smooth and should normally take between one
hour and a few hours. If it takes much longer than that, it's an
indicator that paper writing may be premature.

When answering the questions in this form, use plain, conversational
language - short, simple sentences are preferred!

More information:~https://github.com/computron/scientific-paper-flow

* Indicates required question

1.

Email *

2.

\textbf{What will be the main contributions of this paper?}

\emph{You can list things like data sets, software, new methods, new
models, new equations, a list of new predictions, trained machine
learning models, new knowledge/intuition about a problem, overturning
past ideas, etc. You can (and in most cases should) list more than one
contribution. {[}max: 1500 characters{]}}

*

3.

\textbf{For each of the contributions listed previously, separate them
into the categories below and describe potential figures/tables for that
contribution.}

1. Critical (must-have to be able to submit the paper)

2. Intermediate (would be nice to have and think it is probably
achievable)

3. Stretch (would be amazing to have, would lead to a high-impact
result, but unsure it it's even possible or how to get there if it is)

\emph{You can add additional contributions and ideas from the previous
response if needed for things like "intermediate" and "stretch" goals.}

\emph{For each contribution, try to provide a description of the core
figure(s) / table(s) related to that contribution.}

\emph{{[}max: 1500 characters{]}}

*

4.

\textbf{Who will this paper be useful to, and what kinds of future
studies might cite this work?}

\emph{{[}max: 1500 characters{]}}

*

5.

\textbf{What are some related studies to this paper?}~

Please list at least 5 papers and explain the relation of your paper to
theirs.

\emph{Prefer listing papers outside your direct research group / direct
collaborators. The relation might be directly be on the same topic, but
it could also in terms of having a similar form to your paper (e.g.,
both your paper and the proposed are a data contribution paper even
though they are in different scientific domains).}~

\emph{{[}max: 2000 characters{]}}

*

6.

\textbf{Provide a numbered list of the co-authors of this paper.}

For each co-author, indicate what contribution they might make to help
finish this paper.

\emph{Example contributions include preparation of a figure, processing
of a data set, writing of a section, etc.}

\emph{{[}max: 1500 characters{]}}

*

7.

\textbf{List three or more journals where you might submit this
concept.}

What journal is your top choice?

\emph{{[}max: 1500 characters{]}}

*

8.

\textbf{Use all the responses above to draft a cover letter for this
article.}

Note that it is understood that the cover letter may change depending on
the final research results, however this should be your best guess for
the moment.

* You can address the letter to the editor of one of the journals listed
above

* Your letter should include the (working) title of the paper

* Your letter should contain the contributions of the work (use what you
listed earlier as a guide)

* Your letter can contain some of the citations of related papers

* Your letter should address who will be the audience of the work, and
ideally convince the editor that this paper will have broad readership.

\emph{The letter contents can be typed into the form below. During final
paper submission, this can be used as a guide for your final cover
letter.}

\emph{{[}no maximum length{]}}

*

Bottom of Form

\end{document}
