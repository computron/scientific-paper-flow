\documentclass[]{article}
\usepackage{lmodern}
\usepackage{amssymb,amsmath}
\usepackage{ifxetex,ifluatex}
\usepackage{fixltx2e} % provides \textsubscript
\ifnum 0\ifxetex 1\fi\ifluatex 1\fi=0 % if pdftex
  \usepackage[T1]{fontenc}
  \usepackage[utf8]{inputenc}
\else % if luatex or xelatex
  \ifxetex
    \usepackage{mathspec}
    \usepackage{xltxtra,xunicode}
  \else
    \usepackage{fontspec}
  \fi
  \defaultfontfeatures{Mapping=tex-text,Scale=MatchLowercase}
  \newcommand{\euro}{€}
\fi
% use upquote if available, for straight quotes in verbatim environments
\IfFileExists{upquote.sty}{\usepackage{upquote}}{}
% use microtype if available
\IfFileExists{microtype.sty}{%
\usepackage{microtype}
\UseMicrotypeSet[protrusion]{basicmath} % disable protrusion for tt fonts
}{}
\ifxetex
  \usepackage[setpagesize=false, % page size defined by xetex
              unicode=false, % unicode breaks when used with xetex
              xetex]{hyperref}
\else
  \usepackage[unicode=true]{hyperref}
\fi
\usepackage[usenames,dvipsnames]{color}
\hypersetup{breaklinks=true,
            bookmarks=true,
            pdfauthor={},
            pdftitle={},
            colorlinks=true,
            citecolor=blue,
            urlcolor=blue,
            linkcolor=magenta,
            pdfborder={0 0 0}}
\urlstyle{same}  % don't use monospace font for urls
\setlength{\parindent}{0pt}
\setlength{\parskip}{6pt plus 2pt minus 1pt}
\setlength{\emergencystretch}{3em}  % prevent overfull lines
\providecommand{\tightlist}{%
  \setlength{\itemsep}{0pt}\setlength{\parskip}{0pt}}
\setcounter{secnumdepth}{0}

\date{}

% Redefines (sub)paragraphs to behave more like sections
\ifx\paragraph\undefined\else
\let\oldparagraph\paragraph
\renewcommand{\paragraph}[1]{\oldparagraph{#1}\mbox{}}
\fi
\ifx\subparagraph\undefined\else
\let\oldsubparagraph\subparagraph
\renewcommand{\subparagraph}[1]{\oldsubparagraph{#1}\mbox{}}
\fi

\begin{document}

Top of Form

Scientific Paper Step 5: Final Draft

Step 5 is polishing and refining your nearly submittable draft.~

In this step, you will:

\begin{itemize}
\item
  clean up your text to your quality standard
\item
  finalize your citations for completeness
\item
  perform other improvements you may have self-identified in the
  previous step such as having a more interesting discussion section,
  fine tuning figures, etc.
\item
  double-check for correctness
\end{itemize}

Step 5 usually takes less time than Step 4 (ideally, less than one
week). If it takes longer than that, you may need to consider if you
are~overpolishing.

At the end, you will have finished all 5 steps and have a
ready-to-submit paper!

More information: https://github.com/computron/scientific-paper-flow

* Indicates required question

1.

Email *

Tips for Step 5

Many people actually enjoy this Step 5 because you already have a good
paper, and now it's just about making a good paper great. Furthermore,
most of the adjustments from Step 4 should be optional so the worst-case
scenario is to use your nearly submittable draft from the previous step
and simply make it submittable.

\begin{itemize}
\item
  One main tip - do not get stuck! Some researchers get stuck in this
  final step of the process. They don't know when to stop polishing or
  can't help keep polishing and re-polishing the same paper. This goes
  for text, for figure quality, and also for results. It's important to
  know when to stop. At some point, it's better both for yourself
  \textbf{and} for the research community at large if you stop polishing
  and start working on something else.
\end{itemize}

2.

\textbf{Please complete the following checklist for your final draft.}

Remember to check the boxes below the list!

*~\textbf{All the numbers in the manuscript are correct.} Many
researchers double-check their text and wording multiple times before
paper submission, but don't specifically and separately check just all
the numbers in their text. Many errors in ``final'' manuscripts can be
identified simply by having a separate check for the numbers without
paying attention to any of the text!

\textbf{* All important prior works and research groups are cited.} You
should give credit to prior works where it's due and give readers a
broad perspective of the field and various approaches. Also note that,
fair or not, having a complete reference list can be the difference
between a referee supporting your work and discouraging its publication.

*~\textbf{The abstract and conclusion contain all the "Very Important
Numbers".} For example, in a materials science study, these sections
summarize how many materials were studied, the specific numerical values
of any outstanding property measurements and their conditions, or the
number of candidates recommended for further study.

*~\textbf{The figures/tables and captions alone tell the story of the
paper in the correct order.} A reader should understand what your paper
is about and its major conclusions from the figure and figure captions
alone. Many readers in fact skip reading text altogether and mainly skim
figures and captions along with the abstract/conclusion.

*~\textbf{The acknowledgements are complete.} This includes funding
sources, computing resources, people that helped, and any software you
found helpful.

*

Check all that apply.

All the numbers in the manuscript are correct.

All important prior works and research groups are cited

The abstract and conclusion contain all the "Very Important Numbers"

The figures/tables and captions alone tell the story of the paper in the
correct order

The acknowledgements are complete

Required

3.

\textbf{Please upload the final version of your paper.}

Include the supporting information document if one exists.

Files submitted:

Bottom of Form

\end{document}
