\documentclass[]{article}
\usepackage{lmodern}
\usepackage{amssymb,amsmath}
\usepackage{ifxetex,ifluatex}
\usepackage{fixltx2e} % provides \textsubscript
\ifnum 0\ifxetex 1\fi\ifluatex 1\fi=0 % if pdftex
  \usepackage[T1]{fontenc}
  \usepackage[utf8]{inputenc}
\else % if luatex or xelatex
  \ifxetex
    \usepackage{mathspec}
    \usepackage{xltxtra,xunicode}
  \else
    \usepackage{fontspec}
  \fi
  \defaultfontfeatures{Mapping=tex-text,Scale=MatchLowercase}
  \newcommand{\euro}{€}
\fi
% use upquote if available, for straight quotes in verbatim environments
\IfFileExists{upquote.sty}{\usepackage{upquote}}{}
% use microtype if available
\IfFileExists{microtype.sty}{%
\usepackage{microtype}
\UseMicrotypeSet[protrusion]{basicmath} % disable protrusion for tt fonts
}{}
\ifxetex
  \usepackage[setpagesize=false, % page size defined by xetex
              unicode=false, % unicode breaks when used with xetex
              xetex]{hyperref}
\else
  \usepackage[unicode=true]{hyperref}
\fi
\usepackage[usenames,dvipsnames]{color}
\hypersetup{breaklinks=true,
            bookmarks=true,
            pdfauthor={},
            pdftitle={},
            colorlinks=true,
            citecolor=blue,
            urlcolor=blue,
            linkcolor=magenta,
            pdfborder={0 0 0}}
\urlstyle{same}  % don't use monospace font for urls
\setlength{\parindent}{0pt}
\setlength{\parskip}{6pt plus 2pt minus 1pt}
\setlength{\emergencystretch}{3em}  % prevent overfull lines
\providecommand{\tightlist}{%
  \setlength{\itemsep}{0pt}\setlength{\parskip}{0pt}}
\setcounter{secnumdepth}{0}

\date{}

% Redefines (sub)paragraphs to behave more like sections
\ifx\paragraph\undefined\else
\let\oldparagraph\paragraph
\renewcommand{\paragraph}[1]{\oldparagraph{#1}\mbox{}}
\fi
\ifx\subparagraph\undefined\else
\let\oldsubparagraph\subparagraph
\renewcommand{\subparagraph}[1]{\oldsubparagraph{#1}\mbox{}}
\fi

\begin{document}

Top of Form

Scientific Paper Step 2: Pre-outline

Congratulations on finishing step 1!

In step 2, you will think about:

\begin{itemize}
\item
  what your key results are expected to look like, even if you don't
  have them completed yet. These can change so don't worry about getting
  it perfect at this step. You are looking to clarify the direction of
  the paper
\item
  the order in which you will present your result, which will dictate
  the storytelling of the paper. The order does NOT need to be the order
  in which you did the study, so think carefully about how to best
  communicate the results.
\item
  an estimated timeline for completing the paper. Do your best here and
  don't worry about getting it perfect
\item
  any course corrections you may want or need to make later
\item
  how other people can help - don't underestimate this step!
\end{itemize}

Step 2 usually takes a little bit longer than Step 1, but can usually be
completed in about one day. If it takes much longer than that, it's a
sign that Step 2 is premature.

When answering the questions in this form, use plain, conversational
language - short, simple sentences are preferred!

More information: https://github.com/computron/scientific-paper-flow

* Indicates required question

1.

Email *

Tips for Step 2

Overall, your goal is to keep the momentum going. This means:

\begin{itemize}
\item
  Do not worry about final formatting - e.g., for figures do NOT worry
  about colors, font size, etc. at this stage. In fact, even sketches of
  figures are fine as long as you can articulate a clear vision of that
  figure
\item
  Do not worry about final results or final data. It is OK to submit
  rough drafts or even sketches of figures or tables
\item
  Do not worry about the text at all yet, that will come later.
\item
  Do not worry about the references and citations - these can take a
  long time
\end{itemize}

If you submit Step 2 but don't feel clear in the paper direction, you
can always repeat Step 2 until the vision is relatively clear. It is
completely fine to repeat steps if needed.

2.

\textbf{Upload a single PPT, Word or PDF file containing all the figures
and tables you have prepared so far for this paper.}~

* Include the paper title and author list in the beginning

* Each figure or table should be clearly numbered (e.g., Fig 1, Table 1,
etc.) so you can refer to it in the next question.

* Think carefully about the order of the figures and tables. What should
be presented first, second, third, etc? Note that this order does NOT
need to be the order in which you did the study. For example, it's
likely you will validate a method first and then collect results - but
~some (not all) instances your paper may be better in the opposite order
(results first, validation later).

\emph{Note that the list of figures should include both polished/final
items as well as any rough sketches for figures that may not have final
data or be in final form. Even a doodle of a figure is OK at this stage,
however you should have clarity on what the final figure will look like
and where the data will originate from. If a caption is needed to
understand the figure, please add such a caption.}

\emph{Note that uploading sometimes takes a few minutes, please be
patient.}~

Files submitted:

3.

\textbf{Describe each of the items in the document you uploaded.}

For each item, indicate:

(i) The figure or table number you are referring to ("Fig 1")

(ii) is the figure final ("Final"), Close to final ("Close"), or Rough
("Rough")

(iii) if not final: what steps remain to make this figure final, and an
expected timeline for each step

(iv) reasons why the above timeline(s) may be delayed, and anything you
or a collaborator can do to mitigate the risk of delay.

\emph{Example:}

\emph{-\/-\/-\/-\/-\/-\/-\/-\/-\/-}

\emph{\# Figure 1}

\emph{- Rough}

\emph{- 1 week: finalize determining the workflow steps indicated in
diagram which is just a sketch at this point}

\emph{- 1 week: get buy-in from co-authors on final workflow}

\emph{- 1 day: plot formatting, making it "pretty"}

\emph{- Possible delays: co-authors don't agree on workflow steps; can
schedule a pre-meeting in the next few days to discuss possible ideas
with them in advance.}

\emph{-\/-\/-\/-\/-\/-\/-\/-\/-\/-}

*

4.

\textbf{Apart from figures and tables, describe any other "outputs" of
the paper that may require work.}

For example, this could be data sets, SI information (e.g., crystal
structure files), or software repos that will be shared. For each item,
use the same format as above.

\emph{Example\\
-\/-\/-\/-\/-\/-\/-\/-\/-\/-\\
\# Software (crystal analysis tool)\\
- Close to final.\\
- 3 days: document all the functions\\
- 2 days: full code cleanup\\
- 1 day: take care of proper Python packaging, double check requirements
and installation}

\emph{- Timeline may be delayed if a bug is found after analyzing all
the results; can be mitigated by analyzing partial results in the next
day before full data set analysis.}

\emph{-\/-\/-\/-\/-\/-\/-\/-\/-\/-}

*

5.

\textbf{What conclusions remain uncertain at this stage?}~

For example, you may be waiting for data to decide if a particular
hypothesis is correct or to decide the best way to structure one of your
figures.

\emph{{[}max: 2000 characters{]}}

6.

\textbf{Answer the following questions to ensure that you have the right
trajectory for the remainder of the paper.}

1. Are there initial ideas or approaches that don't seem to be working
out or are taking much longer than expected? If so, what evidence do you
have that things are going to get better over time?

2. New ideas or approaches you should be testing? Note that testing does
not necessarily mean switching.

3. Any changes to the scope of the paper you should consider? For
example, you can refer back to Step 1 of this form which had your
responses for what were critical, intermediate, and ambitious goals.

\emph{{[}max: 1500 characters{]}}

7.

\textbf{List ways in which collaborators of the paper may assist with
unfinished items.}

Do not undervalue thinking about this step, you can finish more quickly
and with more energy if you involve your collaborators!

\emph{{[}max: 1000 characters{]}}

*

Bottom of Form

\end{document}
