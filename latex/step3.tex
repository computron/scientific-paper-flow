\documentclass[]{article}
\usepackage{lmodern}
\usepackage{amssymb,amsmath}
\usepackage{ifxetex,ifluatex}
\usepackage{fixltx2e} % provides \textsubscript
\ifnum 0\ifxetex 1\fi\ifluatex 1\fi=0 % if pdftex
  \usepackage[T1]{fontenc}
  \usepackage[utf8]{inputenc}
\else % if luatex or xelatex
  \ifxetex
    \usepackage{mathspec}
    \usepackage{xltxtra,xunicode}
  \else
    \usepackage{fontspec}
  \fi
  \defaultfontfeatures{Mapping=tex-text,Scale=MatchLowercase}
  \newcommand{\euro}{€}
\fi
% use upquote if available, for straight quotes in verbatim environments
\IfFileExists{upquote.sty}{\usepackage{upquote}}{}
% use microtype if available
\IfFileExists{microtype.sty}{%
\usepackage{microtype}
\UseMicrotypeSet[protrusion]{basicmath} % disable protrusion for tt fonts
}{}
\ifxetex
  \usepackage[setpagesize=false, % page size defined by xetex
              unicode=false, % unicode breaks when used with xetex
              xetex]{hyperref}
\else
  \usepackage[unicode=true]{hyperref}
\fi
\usepackage[usenames,dvipsnames]{color}
\hypersetup{breaklinks=true,
            bookmarks=true,
            pdfauthor={},
            pdftitle={},
            colorlinks=true,
            citecolor=blue,
            urlcolor=blue,
            linkcolor=magenta,
            pdfborder={0 0 0}}
\urlstyle{same}  % don't use monospace font for urls
\setlength{\parindent}{0pt}
\setlength{\parskip}{6pt plus 2pt minus 1pt}
\setlength{\emergencystretch}{3em}  % prevent overfull lines
\providecommand{\tightlist}{%
  \setlength{\itemsep}{0pt}\setlength{\parskip}{0pt}}
\setcounter{secnumdepth}{0}

\date{}

% Redefines (sub)paragraphs to behave more like sections
\ifx\paragraph\undefined\else
\let\oldparagraph\paragraph
\renewcommand{\paragraph}[1]{\oldparagraph{#1}\mbox{}}
\fi
\ifx\subparagraph\undefined\else
\let\oldsubparagraph\subparagraph
\renewcommand{\subparagraph}[1]{\oldsubparagraph{#1}\mbox{}}
\fi

\begin{document}

Top of Form

Scientific Paper Step 3: Full Outline

Congratulations on finishing step 2! In this step you will create the
outline that will develop into your final paper.

You should start step 3 when you already have \textbf{most} (but not
necessarily all) of your main key data and results completed. This is
because in step 3 you are going to begin outlining the text of the
paper, and you need to have some certainty about the results first.

In step 3 you will:

\begin{itemize}
\item
  lay out ALL the exact sections of your paper, forming an outline. This
  includes acknowledgements, data availability, and supplementary
  sections
\item
  figure out what you plan to say in each section~and the order in which
  you will say them. As in the previous step, the order does not need to
  be order in which you did the study.
\item
  create one bullet point for each paragraph that will eventually be in
  the paper, with the bullet providing notes on what to eventually say
  and cite
\item
  include the most updated versions of your figures
\item
  think about parallelizing the next step with co-authors
\end{itemize}

Step 3 usually takes a little bit longer than Step~2, but~can usually be
completed in within a few days. If it takes much longer than that, it's
a sign that Step 3 is premature.~

When answering the questions in this form, use plain, conversational
language - short, simple sentences are preferred!

More information: https://github.com/computron/scientific-paper-flow

* Indicates required question

1.

Email *

Tips for Step 3

As in the previous step, your goal should be to keep the momentum going.
This means:

\begin{itemize}
\item
  Do not worry about writing the ``real'' text (our goal is bullet
  points)
\item
  Do not enter formatted citations (this is time-consuming)
\item
  Do not fine-tune any figures (these often change later in the process)
\end{itemize}

Remember, you don't want perfectionism at this stage, just to keep
momentum. This stage is like an outline for an art project or a
storyboard for a movie. Work in broad strokes. If you are looking for
perfectionism, try to perfect the outline - things like balance, order,
selection of what goes in the main paper versus the supporting
information, etc. If you do this part well, refining the paper in the
next step will be easier - so use your perfectionism towards getting the
perfect outline.~

\textbf{If you are having trouble filling in bullet points}, re-think
the order in which you complete bullets. It is usually easiest if you
add the bullet points for the methods section first as this will be the
easiest to outline. Next will usually be the results section since you
can outline your thoughts around each figure. Last will usually be the
introduction and discussion - these can be difficult since they can
require a lot of literature review.

2.

Attach your paper outline. Instructions for preparing the outline are
below.

1. Put the title and authors of the paper at the top of the document.

2.~List all the section headings to be used in the paper (including
Abstract, Conclusion, etc.). These headings should follow your proposed
journal's guidelines. You do not need to fill in these headings yet,
just list them. Repeat this for subsection headings. This will form an
outline for your paper. If you are unsure of what headings to put, just
look up other / similar papers in the journal you plan to submit to. You
may need to look at a few articles, and also make sure to look at the
same article type as the one you are targeting.

2.~Place all existing figures and tables in the appropriate sections /
subsections. Figure captions should describe what the figure/table is
about as well as the message of the figure. If the figures require
updating in the future (e.g. to the underlying data or to the
presentation), indicate this as comments in your caption.

3. Within each heading and subheading, include \textbf{a single bullet
point for each paragraph-level idea} in that heading. ~For example, the
Introduction may have several bullet points that each list some of the
prior work in a particular topic and together the bullet points outline
the flow of the Introduction. The Results section may have bullet points
that each explains a certain figure, and Methods may have bullet points
that describes each simulation or ML method. The bullets do not need to
be detailed at this point (e.g., one-sentence bullet per paragraph is
fine), but they need to be informative enough that someone should see
the flow of the work. You can also have the bullet points be longer or
contain sub-bullets, containing short notes / very rough writing to help
you remember the things you will eventually polish into text, and
contain notes on potential citations to add. However, do not spend time
writing polished text or inserting formatted citations at this point!

\textbf{Note}: each paragraph in a finished paper is usually
\textasciitilde{}150 words. A paper is normally 3500 - 6000 words. It is
typical you have between 25 - 40 top level bullets, one for each
paragraph.

\emph{Note that it is OK if the paper feels "light" in terms of text at
this stage, or if it feels "messy" with lots of notes of things to talk
about. The notes can be polished into elegant text and elaborated upon
later. The more important thing is to have the figures and overall
structure in a good place, and to have ideas about what you will
eventually talk about.}

*

Files submitted:

3.

\textbf{What conclusions remain uncertain at this stage, and how long
before finalizing results?}

For example, you may be waiting for data to decide if a particular
hypothesis is correct or not. For estimating how long,~indicate
potential delays that might occur, such as computers or equipment going
down, a proposed methodological update not working, ~etc.

\emph{{[}max: 2000 characters{]}}

4.

\textbf{List ways in which collaborators of the paper may assist with
unfinished items.}

\emph{Here's an important tip - if you have co-authors, this is a great
place to parallelize the writing of the paper. If your outline is good
and your co-authors are knowledgeable, it should be straightforward for
them to convert their portion of the outline to text and citations.
You'll be even more successful if you give them a word count to aim for
as well as targets for number of references or figures for their
subsection.}

\emph{{[}max: 1000 characters{]}}

*

Bottom of Form

\end{document}
